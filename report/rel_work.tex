\section{Related Work}
\label{sec:re_work}

\subsection{The Freezing Robot Problem}
The \emph{Freezing Robot Problem} is a classic problem in robotics and consists of finding a balance between being conservative in terms of movement and safety and reach efficiently the goal. If the robot is too conservative and the environment too complex could end up freezing because none of the possible movements would be considered to be safe enough. \\

A classical approach to this problem is to use a planner and try to predict the movement of the surrounding in order to decide the next move, however these approaches usually fail to find a path when the environment is too complex. There are other approaches that tackle the problem from different perspectives. Peter Trautman and Andreas Krause \cite{conf/iros/TrautmanK10} studied and proposed a non-parametric statistical model based on dependent output Gaussian processes that can estimate crowd interaction from data. Chung-Che Yu and Chieh-Chih Wang \cite{6395022} proposed a Learning from Demonstration (LfD) approach proving that these kind of methods are efficient to both avoid collisions and freezing free navigation.

\subsection{Gesture Recognition}
\emph{Gesture Recognition} is an important topic in computer vision and an active line of research for lots of companies. The reason behind this interest in the area is because of the multiple and revolutionary applications to the world of technology, it changes completely the conventional machine input mechanisms (mouse, keyboard and even touch-screens).\\

This technology has being successfully applied in the game industry, changing the gaming experience to a whole new level. Its also has being applied to device control such as tv or intelligent home systems improving the user experience.\\

One of the greatest advances in the topic was the improvement and large scale production of RGB-D cameras which started with the Kinect camera developed by \emph{Microsoft} for the gaming industry. This busted the research in the area. There are countless examples of papers studying the area, a good example of them was presented by Lin Song1, Ruimin Hu, Yulian Xiao and Liyu Gong \cite{LinSong} in 2013 proposing a method for hand segmentation and gesture recognition in real time. Another interesting example was given by Yi Li \cite{6269439} with a method capable of recognizing 9 different gestures with an accuracy between $84\%$ and $99\%$. 
