\section{Conclusions}
\label{sec:conclusions}


In this project we improved the MASHI control system in order to help the teleoperator to operate the robot and also the safety which is really important in the environment the robot is placed. We used a User-Centered Design methodology in order to develop the proposed system. We completed a full cycle and set the things for the next iteration. \\

The cycle started with a proposal of the system, went through a phase of analysing the state of the art, then we designed the system according to the information we had about the robot, the environment and the related literature. Once we implemented the proposed design we tested it with potential users in order to retrieve information and improve the design. \\

Taking into account the problems described in Section \ref{sec:test} we define the following improvements as future work:

\begin{itemize}
	\item
\end{itemize}


%
%
%As a \textbf{future work} we propose some improvements and ideas below:
%
%\begin{itemize}
%	\item Would be useful to display a beep sound when moving backwards, such as trucks do, in order to prevent the people in the surroundings.
%	\item The use of the kinect in the teleoperator side could be used also to add head control.
%	\item The depth camera in the robot could be further exploited by being able to detect humans and behave change its behaviour.
%\end{itemize}
