\section{Introduction}
\label{sec:introduction}

This project is comprehended within the teleoperated robots area. A broad topic recently being pulled further given the increasing need of elderly attention.\\ 

The robot we were working with was MASHI, which is a teleoperated robot with the purpose of guiding in "La Bóbila" social centre in L'Hospitalet del Llobregat. The root consists in an Arduino board embedded to a wheeled structure that resembles the human figure.\\

The environment the robot is going to work in the main entrance of the social centre, where some temporary expositions are hold. MASHI is supposed to welcome the visitors and guide them through the exposition being able to communicate with the visitors.\\

Our project is focused on two different but necessary aspects:
\begin{enumerate}
	\item \textbf{Collisions Avoidance}: We developed a simple but efficient way to avoid collisions using an RGB-D camera attached in the neck of MASHI. Our method tackles the so called \emph{Freezing Robot Problem}.
	\item \textbf{Gesture Control}: Facilitate the teleoperator control of the robot by allowing him/her to use the hands to move the robot. This feature will allow the teleoperator to manage more complex movements such as control head, arms, hands and so on.
\end{enumerate}

The Section \ref{sec:re_work} is a brief compilation of related works and common problems in the area. In the Section \ref{sec:implementation} we explain with all the detail how was carried out the implementation part. The Section \ref{sec:test} contains the experiments with users done in order to test the acceptance and efficiency of out work. In the last section (Section \ref{sec:conclusions}) we expose our conclusions of this project and the possible future work.